\documentclass[11pt,a4paper]{article}
\usepackage{a4wide}
\usepackage{latexsym}
\usepackage{amssymb,amsmath}
\usepackage{amsfonts}
\usepackage{epsfig,graphics,graphicx,graphpap,color}
\usepackage{slashed,xspace,setspace}
\usepackage{caption}
\usepackage{fullpage}
%\usepackage[top=0.8in,right=0.8in,left=0.8in,bottom=1in]{geometry}
\usepackage[top=0.83in]{geometry}
\usepackage{ptdr-definitions}
\usepackage{subfigure}
\usepackage{multirow}
\usepackage{rotate}
\usepackage[sort&compress]{natbib}
\begin{document}

\def\gluino{\mbox{$\tilde g$}\xspace}
\def\mgluino{\mbox{$m_{\tilde g}$}\xspace}
\def\mStop{\mbox{$m_{\tilde t}$}\xspace}
\def\mSbottom{\mbox{$m_{\tilde b}$}\xspace}
\def\mCha{\mbox{$m_{\tilde{\chi}^{\pm}_1}$}\xspace}
\def\mNeu{\mbox{$m_{\tilde{\chi}^{0}_1}$}\xspace}
%\def\eslash{\ensuremath{{\hbox{$E$\kern-0.6em\lower-.05ex\hbox{/}\kern0.10em}}}}
%\def\vecmet{\mbox{$\vec{\eslash}_T$}\xspace} %missing ET vector                  
%\def\vecet{\mbox{$\vec{E}_T$}\xspace} % ET vector                                
%\def\mey{\mbox{$\eslash_y$}\xspace} %missing Ey                                  
\def\nbtags{\mbox{$n_{\cPqb\textrm{-tag}}$}\xspace}
\def\njets{\mbox{$n_{\textrm{jets}}$}\xspace}
\def\mT{\mbox{$m_{\textrm{T}}$}\xspace}
\newcommand{\jptratio}[2]{\mbox{$p_{\textrm{T\,\vline\,#1,#2}}^{\textrm{ratio}}$}\xspace}
\newcommand{\ptb}[1]{\mbox{$p_{\textrm{T}}^{}(b_#1)$}\xspace}
\def\mTW{\mbox{$m_{\textrm{T}2}^W$}\xspace}
\def\HT{\mbox{$H_{\textrm{T}}$}\xspace}
\def\HTratio{\mbox{$H_{\textrm{T}}^{\textrm{ratio}}$}\xspace}
\def\dphi{\mbox{$\Delta\phi(W,l)$}\xspace}
\def\dphimet{\mbox{$\Delta\phi(\ETmiss,j_{1,2})$}\xspace}
\def\ttjets{\mbox{\ensuremath{\cmsSymbolFace{t}\overline{\cmsSymbolFace{t}}}+jets}\xspace}
\def\wjets{\mbox{\ensuremath{W}+jets}\xspace}
\newcommand{\fixme}[1]{\textcolor{red}{FIXME: #1}} % \marginpar{Test}
\newcommand{\tobechecked}[1]{\textcolor{red}{#1}\marginpar{\textcolor{red}{\textbf{X}}}}
\newcommand{\boldStart}[1]{\noindent{\bf{#1}}}
\onehalfspacing
\addtolength{\leftmargin}{-2in}
\thispagestyle{empty}

\section*{Search for natural supersymmetry in 13 TeV collisions with the CMS detector}
%*) 450 words, or 3,000 chars,  1 page PR abstract  1) project title, 2) content of research project, 3) hypotheses, 4) methods, and 5) an explanation indicating what is new and/or special about the project. The
%   language of the PR abstracts should be comprehensible to non-specialist audiences and contain as few technical/specialist terms as possible;

For five years, particle physicists at the Large Hadron Collider (LHC) at CERN have been analyzing collisions of protons 
that are are collided at 99.9999968\% of the speed of light or, equivalently, at an energy of 8 \TeV.

In these high energetic reactions, a new particle was discovered that had properties just as predicted for the long sought Higgs boson. 
Measurements of decay properties confirmed this suspicion and the two major experimental collaborations, ATLAS and CMS, proudly announced that milestone on July 4${}^{\textrm{th.}}$, 2012.

Theoretically, the Higgs boson is a challenge because of the special way it interacts with the other particles. 
This famous ``Higgs meachanism'' provides masses to the fundamental particles, 
but when the relations are turned around, the Higgs boson receives large corrections to its mass from those particles in turn. 
The net sum of these corrections needs to be 
stabilized according to widespread belief, or otherwise the Higgs boson would be almost infinitely heavy. 
Interestingly, the corrections have different sign for force carrier particles, such as the photon, which is the particle of visible light, and matter particles, such
as the electron. If a property of nature ensured that there was the same number of particles for each of those species, then the corrections cancelled themselves.
Supersymmetry, a theoretical framework invented in the 1970s, provides exactly this mechanism. 
It not only stabilizes the Higgs boson mass, it also predicts that there is a stable particle
that does hardly interact with other matter. 
Those particles would therefore fill the universe, interact only gravitationally and leave almost no trace otherwise. Is it a coincidence 
that our universe seems to contain 63\% of so called dark matter which exactly fits this prescription? 
If this is a hint for supersymmetry, how can we uncover it in collision data? 

This proposal addresses these questions by analyzing LHC collision data taken by the CMS experiment until 2018. By then, enough data will have been accumulated to decide either way.
The proposal is a joint effort by the University of Ghent (UGent, Prof. D. Dobur) and the Institute for High Energy Physics in Austria (HEPHY, R. Sch\"ofbeck) who combine their strengths in analyzing
collisions where also electrons or muons, collectively called leptons, are created. 
Event samples with exactly one lepton or a pair of leptons with the same charge have very different properties. Hence, different analysis strategies are required for the two cases.
Combining these channels, however, promises a tremendous gain in sensitivity. Therefore, we plan a joint analysis strategy comprising same-charge lepton pairs led by UGent
and the single-lepton channel led by HEPHY. In this way, we will either uncover supersymmetry as a fundamental property of nature or constrain it so severely that it loses most of its theoretical appeal.

\end{document}
