\documentclass[11pt,a4paper]{article}
\usepackage{a4wide}
\usepackage{latexsym}
\usepackage{amssymb,amsmath}
\usepackage{amsfonts}
\usepackage{epsfig,graphics,graphicx,graphpap,color}
\usepackage{slashed,xspace,setspace}
\usepackage{caption}
\usepackage{fullpage}
%\usepackage[top=0.8in,right=0.8in,left=0.8in,bottom=1in]{geometry}
\usepackage[top=0.83in]{geometry}
\usepackage{ptdr-definitions}
\usepackage{subfigure}
\usepackage{multirow}
\usepackage{rotate}
\usepackage[sort&compress]{natbib}
\begin{document}

\def\gluino{\mbox{$\tilde g$}\xspace}
\def\mgluino{\mbox{$m_{\tilde g}$}\xspace}
\def\mStop{\mbox{$m_{\tilde t}$}\xspace}
\def\mSbottom{\mbox{$m_{\tilde b}$}\xspace}
\def\mCha{\mbox{$m_{\tilde{\chi}^{\pm}_1}$}\xspace}
\def\mNeu{\mbox{$m_{\tilde{\chi}^{0}_1}$}\xspace}
%\def\eslash{\ensuremath{{\hbox{$E$\kern-0.6em\lower-.05ex\hbox{/}\kern0.10em}}}}
%\def\vecmet{\mbox{$\vec{\eslash}_T$}\xspace} %missing ET vector                  
%\def\vecet{\mbox{$\vec{E}_T$}\xspace} % ET vector                                
%\def\mey{\mbox{$\eslash_y$}\xspace} %missing Ey                                  
\def\nbtags{\mbox{$n_{\cPqb\textrm{-tag}}$}\xspace}
\def\njets{\mbox{$n_{\textrm{jets}}$}\xspace}
\def\mT{\mbox{$m_{\textrm{T}}$}\xspace}
\newcommand{\jptratio}[2]{\mbox{$p_{\textrm{T\,\vline\,#1,#2}}^{\textrm{ratio}}$}\xspace}
\newcommand{\ptb}[1]{\mbox{$p_{\textrm{T}}^{}(b_#1)$}\xspace}
\def\mTW{\mbox{$m_{\textrm{T}2}^W$}\xspace}
\def\HT{\mbox{$H_{\textrm{T}}$}\xspace}
\def\HTratio{\mbox{$H_{\textrm{T}}^{\textrm{ratio}}$}\xspace}
\def\dphi{\mbox{$\Delta\phi(W,l)$}\xspace}
\def\dphimet{\mbox{$\Delta\phi(\ETmiss,j_{1,2})$}\xspace}
\def\ttjets{\mbox{\ensuremath{\cmsSymbolFace{t}\overline{\cmsSymbolFace{t}}}+jets}\xspace}
\def\wjets{\mbox{\ensuremath{W}+jets}\xspace}
\newcommand{\fixme}[1]{\textcolor{red}{FIXME: #1}} % \marginpar{Test}
\newcommand{\tobechecked}[1]{\textcolor{red}{#1}\marginpar{\textcolor{red}{\textbf{X}}}}
\newcommand{\boldStart}[1]{\noindent{\bf{#1}}}
\onehalfspacing
\addtolength{\leftmargin}{-2in}
\thispagestyle{empty}

\section*{Abstract}
During the 2011/2012 data-taking period of the Large Hadron Collider (LHC), the CMS and ATLAS experiments each recorded proton-proton collisions amounting to an integrated luminosity of 25~fb$^{-1}$ at a centre-of-mass energy of 7 and 8 \TeV and discovered a new boson in this dataset, most probably a scalar particle. 
Measurements of the decay branchings are compatible with the predictions for the Higgs boson of the Standard Model (SM).

Another salient feature of the data is the absence of any supersymmetric (SUSY) signature, resulting in lower limits for the masses of many strongly-interacting SUSY particles of the order of 1 \TeV. % for many strongly-interacting SUSY masses.
Can SUSY still provide the stabilization of the quantum corrections to the Higgs boson mass according to the long advertised scheme,  where fermionic and bosonic loops cancel? 
%A cancellation of fermionic and bosonic loops in the Higgs mass calculation is possible, albeit with some fine-tuning.
%The cancellation of bosonic and fermionic loops 
And is the discovered scalar particle not the SM Higgs boson, but in fact, the lightest SUSY Higgs particle?

At present, the SUSY conjecture towards these questions is still open and will be submitted to stringent tests with data from the LHC runs at 13~\TeV starting in 2015.
In this proposal we focus on ``natural'' SUSY, which condenses the idea that this new symmetry of nature preserves the quantum Higgs boson mass.
It has become a priority for the experimental collaborations at the future high energy run and a central pillar of the general idea of SUSY as a theory of fundamental interactions. 

The LHC is the unique place to tackle these questions experimentally because of its unprecedented collision energies.
Our aim is to use CMS data to search for gluinos, new particles predicted by natural SUSY and with masses in the reach of the next LHC run at a centre-of-mass energy of 13 \TeV.
The dataset will comprise 100 fb${}^{-1}$ at 13 \TeV collected until 2018. 
Technically, we plan a two-fold research project to search for signals of gluino-initiated stop or sbottom production, comprising final states with a single lepton and with two leptons of the same charge.
%In summary, 13~\TeV LHC data will open up a new unexplored territory towards the high gluino masses.
We expect the single-lepton channel to play the leading role in probing the highest SUY particle masses, while the power of the same-charge channel will be most prominent for compressed SUSY mass spectra,
where the total amount of energy released in the decay is comparably small.
This bundles the strengths of the UGent CMS analysis group (Prof. D. Dobur) and the HEPHY CMS analysis group (R. Sch\"ofbeck) into a unique tandem search for natural SUSY, while
the two components are complementary by design. We embrace more than half of the SUSY signal events resulting from gluino initiated production of third generation squarks and 
in this way, we will finally settle the question whether or not natural SUSY is realized in nature. 
\end{document}
