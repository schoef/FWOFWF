\documentclass[11pt,a4paper]{article}
\usepackage{a4wide}
\usepackage{latexsym}
\usepackage{amssymb,amsmath}
\usepackage{amsfonts}
\usepackage{epsfig,graphics,graphicx,graphpap,color}
\usepackage{slashed,xspace,setspace}
\usepackage{caption}
\usepackage{fullpage}
%\usepackage[top=0.8in,right=0.8in,left=0.8in,bottom=1in]{geometry}
\usepackage[top=0.83in]{geometry}
\usepackage{ptdr-definitions}
\usepackage{subfigure}
\usepackage{multirow}
\usepackage{rotate}
\usepackage[sort&compress]{natbib}
\begin{document}

\def\gluino{\mbox{$\tilde g$}\xspace}
\def\mgluino{\mbox{$m_{\tilde g}$}\xspace}
\def\mStop{\mbox{$m_{\tilde t}$}\xspace}
\def\mSbottom{\mbox{$m_{\tilde b}$}\xspace}
\def\mCha{\mbox{$m_{\tilde{\chi}^{\pm}_1}$}\xspace}
\def\mNeu{\mbox{$m_{\tilde{\chi}^{0}_1}$}\xspace}
%\def\eslash{\ensuremath{{\hbox{$E$\kern-0.6em\lower-.05ex\hbox{/}\kern0.10em}}}}
%\def\vecmet{\mbox{$\vec{\eslash}_T$}\xspace} %missing ET vector                  
%\def\vecet{\mbox{$\vec{E}_T$}\xspace} % ET vector                                
%\def\mey{\mbox{$\eslash_y$}\xspace} %missing Ey                                  
\def\nbtags{\mbox{$n_{\cPqb\textrm{-tag}}$}\xspace}
\def\njets{\mbox{$n_{\textrm{jets}}$}\xspace}
\def\mT{\mbox{$m_{\textrm{T}}$}\xspace}
\newcommand{\jptratio}[2]{\mbox{$p_{\textrm{T\,\vline\,#1,#2}}^{\textrm{ratio}}$}\xspace}
\newcommand{\ptb}[1]{\mbox{$p_{\textrm{T}}^{}(b_#1)$}\xspace}
\def\mTW{\mbox{$m_{\textrm{T}2}^W$}\xspace}
\def\HT{\mbox{$H_{\textrm{T}}$}\xspace}
\def\HTratio{\mbox{$H_{\textrm{T}}^{\textrm{ratio}}$}\xspace}
\def\dphi{\mbox{$\Delta\phi(W,l)$}\xspace}
\def\dphimet{\mbox{$\Delta\phi(\ETmiss,j_{1,2})$}\xspace}
\def\ttjets{\mbox{\ensuremath{\cmsSymbolFace{t}\overline{\cmsSymbolFace{t}}}+jets}\xspace}
\def\wjets{\mbox{\ensuremath{W}+jets}\xspace}
\newcommand{\fixme}[1]{\textcolor{red}{FIXME: #1}} % \marginpar{Test}
\newcommand{\tobechecked}[1]{\textcolor{red}{#1}\marginpar{\textcolor{red}{\textbf{X}}}}
\newcommand{\boldStart}[1]{\noindent{\bf{#1}}}
\onehalfspacing
\addtolength{\leftmargin}{-2in}
\thispagestyle{empty}

\section*{Suche nach nat\"urlicher Supersymmetrie in 13 TeV Kollisionen mit dem CMS Detektor}
%*) 450 words, or 3,000 chars,  1 page PR abstract  1) project title, 2) content of research project, 3) hypotheses, 4) methods, and 5) an explanation indicating what is new and/or special about the project. The
%   language of the PR abstracts should be comprehensible to non-specialist audiences and contain as few technical/specialist terms as possible;

%Since five years, particle physicists at the Large Hadron Collider at CERN analyze collisions of protons 
%that are are collided at 99.9999968\% of the speed of light or, equivalently, at an energy of 8 \TeV.
Seit f\"unf Jahren untersuchen Wissenschaftler am LHC Proton-Proton Kollisionen mit einer Schwerpunktsenergie von 8 \TeV oder 99.9999968\% der Lichtgeschwindigkeit.

%In this high energetic reactions, a new particle was discovered that had properties just as predicted for the long sought Higgs boson. 
%Measurements of decay properties confirmed this suspicion and the two major experimental collaborations, ATLAS and CMS, proudly announced the result on July 4${}^{\textrm{th.}}$, 2012.

In diesen hochenergetischen Reaktionen wurde ein Teilchen entdeckt, das genau die Eigenschaften des lange gesuchten Higgs Bosons hat.
Messungen der Zerfallswahrscheinlichkeiten best\"atigten den Verdacht und mit viel Stolz pr\"asentierten die beiden grossen Kollaborationen, ATLAS und CMS,
des Ergebnis gemeinsam am 4. Juli 2012.

%Theoretically, the Higgs boson is a challenge because of the special way it interacts with the other particles. 
%This famous ``Higgs meachanism'' provides masses to the other fundamental particles, 
%but when the relations are turned around, the Higgs boson itself receives large corrections to its mass from those particles. 
%The net sum of these corrections needs to be 
%stabilized according to widespread belief, or otherwise the Higgs boson would be almost infinitely heavy. 
%Interestingly, the corrections go in different directions for force carrier particles, such as the photon, which is the particle of visible light, and matter particles, such
%as the electron. If a property of nature ensured that there was the same number of particles for each of those species, then the corrections would cancel among themselves.
%Supersymmetry, a theoretical framework invented in the 1970s, provides exactly this mechanism. 
%It not only stabilizes the Higgs boson mass, it also predicts that there is a stable particle
%that does hardly interact with other matter. 
%Those particles would therefore fill the universe, interact only gravitationally and leave almost no trace otherwise. Is it a coincidence 
%that our universe seems to contain 63\% so called dark matter that exactly fits that prescription? 
%If this is a hint for supersymmetry, how can we uncover it in collision data? 
In der Theorie ist das Higgs Boson ein heikles Teilchen aufgrund der Art seiner Wechselwirkung mit Materie.
Der ber\"uhmte ``Higgs Mechanismus'' verleiht den Elementarteilchen ihre Masse. Bestimmt man aber die Masse des Higgs Bosons selbst,
so zeigt sich, dass auch umgekehrt die Masse des Higgs Bosons von den anderen Teilchen beeinflusst ist. Diese Beitr\"age sind sehr gross und
es wird angenommen, dass sie in unserer Naturbeschreibung stabilisert werden m\"ussen. Interessanterweise sind die Beitr\"age von Kraftteilchen (zB. dem Lichtteilchen, dem Photon)
und den Materieteilchen (zB. dem Elektron) entgegengesetzt. G\"abe es nun eine Eigenschaft der Natur welche erzwingt, dass es gleich viele Teilchen beider Sorten gibt, dann
w\"urden sich die Beitr\"age wegheben. Tats\"achlich erreicht Supersymmetie, ein hypothetischer Rahmen der in den 1970er Jahren geschaffen wurde, genau das. 
Sie stabilisert nicht nur die Masse des Higgs Bosons sondern sagt auch ein schwach wechselwirkendes, aber stabiles Teilchen voraus. Solche Teilchen w\"urden
das Universum durchdringen und gravitativ wechselwirken, sonst aber kaum in Erscheinung treten. Ist es ein Zufall, dass astronomische Messungen die ``Dunkle Materie'' mit 63\% veranschlagen,
welche genau dieser Beschreibung entspricht? Und falls das ein Hinweis auf Supersymmetrie ist, wie entdecken wir sie an Beschleunigerexperimenten?

%This proposal addresses these questions by analyzing LHC collision data taken by the CMS experiment until 2018. By then, enough data will have been accumulated to decide either way.
%The proposal is a joint effort by the University of Ghent (UGent, Prof. D. Dobur) and the Institute for High Energy Physics in Austria (HEPHY, R. Sch\"ofbeck) who combine their strengths in analyzing
%collisions were also electrons or muons, collectively called leptons, are created. 
%Event samples with exactly one or two and more leptons of the same charge have very different properties. Therefore, different analysis strategies are required for the two cases.
%Combining these channels, however, promises a tremendous gain in sensitivity. Therefore, we plan a joint analysis strategy comprising same-charge lepton pairs lead by UGent
%and the single-lepton channel lead by HEPHY. In this way, we will either exclude or uncover this new fundamental property of nature within the next three years.
Dieser Antrag versucht diese Fragen mit dem CMS Datensatz bis zum Jahr 2018 zu entscheiden. Der Antrag wird gemeinsam von der Universit\"at Gent (Prof. D. Dobur, UGent) und dem
Institut f\"ur Hochenergiephysik (HEPHY, R.Sch\"ofbeck) der OeAW eingebracht. Kollisionsereignisse mit entweder genau einem Lepton (ein Elektron oder ein Myon) oder
einem Leptonpaar derselben Ladung haben sehr unterschiedliche Eigenschaften. Deshalb sind komplement\"are Strategien bei der Hintergrundabsch\"atzung erforderlich.
Kombiniert man die Kan\"ale aber, so zeigt sich eine starke Verbesserung der Sensitivit\"at. Daher planen wir eine gemeinsame Analyse welche die Kan\"ale mit einem
Lepton (HEPHY) und mit einem Leptonpaar gleicher Ladung (UGent) zusammenf\"uhrt. Damit k\"onnen wir die Supersymmetrie als grundlegende Eigenschaft der Natur nun entweder entdecken oder 
aber soweit einschr\"anken dass sie v\"ollig unplausibel wird.

\end{document}
